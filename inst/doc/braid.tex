\documentclass[a4paper]{article}

%\VignetteIndexEntry{Introduction to braid}
%\VignettePackage{braid}

% Definitions
%\newcommand{\slan}{{\tt S}}
%\newcommand{\rlan}{{\tt R}}
\newcommand{\braid}{{\tt braid}}
\newcommand{\code}[1]{{\tt #1}}
\setlength{\parindent}{0in}
\setlength{\parskip}{.1in}
\setlength{\textwidth}{140mm}
\setlength{\oddsidemargin}{10mm}

\title{Introduction to the \braid{} package for creating PDF reports in R}
\author{Andrie de Vries}

\usepackage{Sweave}
\begin{document}

\maketitle

\braid{} is a package that makes it easy to create PDF report in R without having to learn SWeave or Latex.  It creates a few high level R functions to control the creation and compilation of LaTEX statements:

Create a new braid: 
\begin{itemize}
	\item \code{as.braid()}
\end{itemize}

Report writing commands:
\begin{itemize}
	\item \code{braid\_heading()}
	\item \code{braid\_write()}
	\item \code{braid\_plot()}
\end{itemize}

Save results to file:
\begin{itemize}
	\item \code{braid\_save()}
\end{itemize}

Compile to PDF:
\begin{itemize}
	\item \code{braid\_latex\_outline()}
	\item \code{braid\_compile()}
\end{itemize}
	

\section{Introduction}

There are several report writing packages that uses markup code to create reports in R.  This includes SWeave, Brew and ODFweave.

To use these packages, the user will write markup code with embedded R statements.  For example, in SWeave, the user writes LaTEX statements which are then weaved to split the markup into R statements and Latex statements.

Braid approaches this problem from the other direction.  In other words, the user always write R code.  The results of this code are then wrapped in Latex statements and saved to a .tex file.

Finally, braid provides a wrapper function around \code{\link{tools::tex2dvi}} that will "compile" the document into PDF.

In principle, the package can be extended to also support compilation to HTML or ODF formats, but at the moment only PDF via Latex is supported.

Why did I call it \braid{}?  Because braiding extends the visual image of weave. R code is braided first into Latex and then to PDF. 


\section{Setting up a braid}

First, set up a braid object.  The following code sets up a braid object.  A braid object sets up a text file that will contain the report code.

\begin{Schunk}
\begin{Sinput}
> library(braid)
> path <- file.path("f:", "git", "braid", "test")
> latex_path <- file.path(path, "latex")
> graph_path <- file.path(latex_path, "graphics")
> sinkfile <- file.path(latex_path, "braid_test.tex")
> outline_file <- file.path(latex_path, "Outline.tex")
> content_file <- "Content.tex"
> pdf_file <- "Outline.pdf"
> b <- as.braid(pathLatex = latex_path, pathGraphics = graph_path, 
+     outputFilename = file.path(latex_path, content_file))
\end{Sinput}
\end{Schunk}

\section{Create some report output}

Now, lets create some headings, text and graphics to go into the report.

There are three essential report elements in a typical report:

\begin{itemize}
	\item \code{braid\_heading()}
	\item \code{braid\_write()}
	\item \code{braid\_plot()}
\end{itemize}


\begin{Schunk}
\begin{Sinput}
> braidHeading(b, "This is a test")
> braidWrite(b, "This should be a normal paragraph.")
> braidWrite(b, "Let's create a simple plot.")
> library(ggplot2)
> p <- ggplot(iris, aes(x = Sepal.Length, y = Sepal.Width)) + geom_point(alpha = 0.5)
> braidPlot(b, p)
\end{Sinput}
\end{Schunk}

\section{Save and compile results}

To save and compile:

\begin{Schunk}
\begin{Sinput}
> braidSave(b)
> braid_latex_outline(outputFilename = outline_file, content_filename = content_file, 
+     title = "Test", author = "I am the author")
> braidCompile(outline_file)
\end{Sinput}
\end{Schunk}

If you have not received any error messages, the pdf output should now be ready to 


\section{Final thoughts}

The last word....


% Start a new page
% Not echoed, not evaluated
% ONLY here for checkVignettes so that all output doesn't
% end up on one enormous page

\end{document}




